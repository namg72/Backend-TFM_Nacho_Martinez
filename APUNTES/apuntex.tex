.- Iniciamos el proyecto node => npm init -y

.- Instalamos las de pendencias de express y de mySQL

.- requerimos las variables de entorno  dotenv y usammos su metodo config()

.- Nos creamos un archivo app.js

********************************************* app.js ***********************************************************

.- Requerimoms expess y lo iniciamos:
        
.- Requerimoms las variables de entorno => require('dotenv').config()
   nos creamos un archivo .env en la raiz del proyecto y definmos las variables de entorno 

********************************************* DBConfig ***********************************************************   

.- Creamos una carpeta DBConfig y dentro de ella un archivo .js donde definiremos la conexion con la basse de datos

.- requerimos las variables de entorno  dotenv y usammos su metodo config()

.- Creamos un objeto conexion del tipo msql y usamos el metodo .createConexión que recibe los parametros del conexion

.- Con el objeto conexion usamos el metodo conect. Controlamos el error

.- Exportamos el objeto conexion

********************************************* app.js ***********************************************************

.- Escuchamos la conexion a la base de datos con el metodo liste que recibe como parametos el puerto del servidor y 
   como segundo parametro puede recibir una función
        app.listem(puertoServidor, ()=>{})

 
********************************************* CARPETA ROUTES ***********************************************************

En esta carpeta vamos a definr tantos archivos como rutas tengamos. En este caso cremaos dos para las tablas de user y proeuctos

.- Nos traemos express y utilizamos su metod Router()

.- Importamos los controladores de las rutas

.- con el objeto route utilizamos los metodos get/post/delete/update recibiendo como prime parametro la ruta y luego el controladores

._ Exportamos el archivo routes que lo usaremos en la app principal

********************************************* CARPETA CONTROLLER ***********************************************************

En esta carpeta crearemos los metodos para las rutas de cada empoint

.- Primero Nos creamos un objeto controlador vacio

.- Nos traemos la conexion

.- Creammos los metodos CRUD

.- Exportamos los metodos


********************************************* app.js ***********************************************************

importamos la rutas y le decimos a la apliacion que las use

Creamos un middleware para indicarle a express que vamos a trabajar con json
         appe.use(express.json())


================================================V A L I D A C I O N============================================

vamos a validar los datos incluidos en la petición de insertar usuarios mediante middelwares
instalamos el paquete expre-validator 
utilizamos la funcion check del paquete express-validator
 En la ruta antes del controllador ponemos un array con los middelwares que requiramos para validar
dentro del check ponemos primero el nombre del campo que queremos validar, luego un mensaje de error por si no pasa, 
y luego la función de validacion
         
El check directamente no para el ingreso de datos sino que recoge los errores  en un array y los inyecta en el request
esos errores los tenemos que controlar en el controller mediante el meetodo validationResult de express-validator
En el controlador tenemos que importar metodo validationResult de express-validator     





=============================================E N C R I P T A C I O N============================================

CONTROLLER USER. 

Vamos a guardar la congtraseña encriptada en al base de datos

 Instalamos el paquete npm i bryptjs y lo importamos.

y tenemos que encriptar el password ante de insertarlo

tenemos que generar un salt llamamos al paquete bycrity y llamamos al paquete getSaltSync() y

 no creamos una varialble que le vamos a iguar al paquete bycrit con el metodo hasSync() que recibe dos parametros, el salt
 y la contraseña que queremos encriptar


 *=============================================L  O  G  I  N============================================


 Nos creamos una funcion de login
   
   .- Validamos que el nombre de usuario y contraseña sean correctos

   .- Creamos las variable que recojemos del body
 
   .- Nos traemos idUsuairo y la contraseña del nombre de usuario que le pasamos, si no recibimos nada es que hubo un error

   .- Comprobamos que la contraseña que recibimos es igual a la contraseña encirptada que tenemos en la bd
      Para ellos usamos el metodo Sincrono "compareSync()" para que el programa se espere a que se ejecute la linea
       y le pasamos primero la contraseña sin escritar que envia en cliente y llegar por el req,  y luego la 
       contraseña encriptada que esta daentro de los rows en la primera posicion con el campo "password" 
       Y si es true pues el login es exitios

       Para ello nos creamos una variable que se igual al metodo anterior para que coger si es true o false. Si es true
       creamos el JWT que definilmos en el helpers

    .- Si todo va bien le enviamos el  token que nos devuleve como promesa la funcion crearToken definida en los helpesrs
       y si va mal en le catch  le enviamos el mensaje de erro definido en la propia función

     
     ,-  En las rutas definimos la ruta para el login  





     /*========================================== C R E A R  J  W  T=================================================

      Vamos a crear esta función como helper por si tuvieramos que reutilizarla en este u otro proyecto

      Instalamos el modulo jsonwebtoken y lo importamos


     Funcion para crear el token, le pasamos el id del iusuario por parametro y nos devueve una promesa
    
     .- Primero tenemos que construir el payload del jwt. esto lo almacenamos en una variable 
        En el payload le indicamos la información (no comprometida) del usuario que definamos nosotros, en este caso
        solo le vamos a pasar el id del usuario
    
      .- Utilizamos la función sign() del objeto jwt dle jwt que hemos importado. Recibe dos parametros, el payload y la firma
         La firma la cramos nosotros, tiene que ser largay  con caracteres especiales para ser sergura, la  vamos a guardar como
         variable de entorno. Tambien reacibe unas opciones que van como siemore entre {}. le vamo a poner el tiempo de expiracion
    
         Express no trabaja con variables de entorno por lo que tenemos que importar ese paquete  (mpm i dotenv)
    
    
       .- Sing() recibe varios parametros;
       
            1_  El payload, 
            2.- La firma creada
            3.- Opciones:
                    Las opciones son objeto y tiene atributos como por eemplo el tiempo de expiracion del token. 
                    Si no le indicamos nada el token no expirará nunca
            4,. Un callback que recibe como parametros un error y el token y es es que resuelve la promesa
                Si hay un error lo definmos en el reject
                Y si todo va bien devolvemos el tokden en el resolve
           
    
          
        .- Por ultimo lo exportamos y lo usamos en el controlador
    
    
    
    
    
    *
